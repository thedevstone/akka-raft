 % -*- root: ../main.tex -*-
\chapter{RAFT}
	Come abbiamo visto nel capitolo precedente la struttura di una macchina a stati replicata è generalmente molto semplice. Questo è dato dal fatto che ogni singola state machine può essere decomposta in singoli \textbf{moduli} dedicati a uno \textbf{specifico scopo}.
	\begin{itemize}
		\item{\textit{Log Module}}
		\item{\textit{State Machine Module}}
		\item{\textit{Consensus Module}}
	\end{itemize}
	I moduli sono \textbf{semi-dipendenti} ossia comunicano tra di loro attraverso chiamate a procedure standard. Per esempio il modulo di log potrebbe esporre funzionalità dedicate all'appending o commiting di una serie di comandi, o ancora la state machine potrebbe mettere a disposizione un metodo per l'esecuzione di un determinato comando.\\
	In questo capitolo andremo a soffermarci maggiormente sul core-module di ogni SMR ossia il \textbf{modulo di consenso}; capiremo quali sono le sfide che un algoritmo di consenso deve affrontare e vedremo in dettaglio un'implementazione di un famoso algoritmo di consenso, \textbf{RAFT}.

	\section{Approcci al Consenso}
    % -*- root: ../../main.tex -*-
Esistono generalmente due approcci al problema del consenso:
\begin{itemize}
  \item{\textbf{Symmetric | Leader-less:}}
  In questa tipologia i server hanno tutti lo \textbf{stesso ruolo} e tutti lo stesso potere decisionale in qualsiasi momento.\\
  I \textbf{client} possono quindi \textbf{contattare} e fare richiesto verso \textbf{qualunque server} presente all'interno del cluster. 
  \item{\textbf{Asymmetric | Leader-base:}}
  In qualsiasi momento i server non hanno \textbf{mai} lo \textbf{stesso ruolo} e lo stesso potere decisionale. Esiste infatti sempre un server che si trova \textbf{in carica}, questo particolare server viene denominato \textbf{Leader}. Il leader gestisce solitamente tutte le operazioni del cluster, mentre gli altri server sono denominati \textbf{Servant} e portano a compimento le richieste del leader.\\
  In questo tipo di sistema i \textbf{client} possono \textbf{comunicare solamente con il leader}; nel caso in cui un client contatti un server diverso dal leader viene immediatamente \textbf{rediretto} al leader corrente.
\end{itemize}
Vedremo che RAFT implementa la seconda versione di consenso; questo garantisce all'algoritmo di essere estremamente più semplice dei suoi predecessori leader-less garantendo comunque lo stesso grado di safety.
  \section{L'algoritmo PAXOS}
    % -*- root: ../../main.tex -*-
Paxos è un algoritmo di consenso, non resistente a Byzantine Fault, sviluppato verso la fine degli anni ottanta e pubblicato nel 1998. Esso ha ricoperto il ruolo di algoritmo standard per decenni, venendo utilizzato in diversi ambienti, compreso quello didattico, poichè rappresenta un'ottima alternativa al commit a due fasi e a quello a tre fasi che si affidano entrambi ad un unico coordinatore che rappresenta un single point of failure. Esistono molte varianti di Paxos, ma la più usata si chiama \textit{Single Decree Paxos}.

In Paxos, i nodi possono assumere contemporaneamente uno o più dei seguenti ruoli:
\begin{itemize}
	\item \textbf{Proposer:} propone un valore su cui accordarsi a tutti gli acceptor;
	\item \textbf{Acceptor:} riceve le proposte dei proposer, decide se accettarle o meno e inoltra la propria risposta ai learner;  
	\item \textbf{Learner:} sulla base delle risposte ricevute, determina come valore vincitore quello scelto dalla maggioranza degli acceptor.
\end{itemize}


  \begin{figure}[H]
    \centering
    \includegraphics[width=0.90\columnwidth]{paxos/proposeAcceptLearn.png}
    \caption{Esempio di configurazione dei ruoli con due \textbf{proposer}, tre \textbf{acceptor} e un \textbf{learner} che sono rappresentati come nodi distinti per semplicità: nella realtà un nodo può ricoprire più di un ruolo alla volta.}
    \label{fig:figure 5}
  \end{figure}


Qualsiasi criterio si scelga per far decidere agli acceptor quale valore accettare, si incorre in casi di incorrettezza a meno di non utilizzare un approccio in due passi. 
Il raggiungimento del consenso si sviluppa quindi in due fasi.

\begin{enumerate}
	\item \textbf{Proposal:} in questa fase i proposer inviano una richiesta di \textit{prepare(n)} agli acceptor, dove \textit{n} è un \textit{proposal number} scelto in maniera tale che sia maggiore di ogni altro numero incontrato fino a quel momento. Gli acceptor ricevono le proposte e, per ognuna, controllano se il valore corrispondente è il maggiore in cui si siano imbattuti fino a quel momento: in caso affermativo, rispondono alla proposta impegnandosi a non accettare proposte precedenti, altrimenti la ignorano. Se un proposer riceve una risposta alla propria proposta dalla maggiorparte degli acceptor, passa alla seconda fase.
	
	\item \textbf{Accept:} in questa fase, il proposer invia il \textit{proposal number} e il \textit{value} tramite una richiesta \textit{accept(n,v)} rivolta a tutti gli acceptor. Se il proposer si vede ritornare come risposta il proprio \textit{proposal number} valore, allora la maggioranza degli acceptor ha concordato sul valore proposto, che quindi sarà quello scelto. Nel caso in cui ciò non accada, il proposer si vedrà arrivare un \textit{proposal number} maggiore del poprio, di conseguenza ricomincerà l'iter, tornando all'inizio della prima fase.
\end{enumerate}

  \begin{figure}[H]
    \centering
    \includegraphics[width=0.90\columnwidth]{paxos/proposersAcceptorsInteraction.png}
    \caption{Schema riassuntivo delle interazioni tra proposer e acceptor, che non tiene conto, per semplicità, dei learner}
    \label{fig:figure 6}
  \end{figure}

L'algoritmo Paxos non è così semplice come potrebbe sembrare a una prima lettura, ma nasconde una serie di dinamiche complesse che non verranno analizzate in questo elaborato poichè è sufficiente un'infarinatura sul funzionamento generale di Paxos per capire in cosa differisce RAFT. 

Nonostante sia ancora largamente usato, l'algoritmo Paxos presenta dei difetti che RAFT ha provato a colmare:
\begin{itemize}
	\item \textbf{Difficoltà di comprensione:} la parte teorica e le dinamiche dell'algoritmo sono di difficile comprensione e spiegazione, come evidenziato anche dagli esperimenti condotti dagli autori di RAFT;
	
	\item \textbf{Difficoltà di implementazione:} anche una volta assimilata la parte teorica, è difficile applicarla. Quando si scende nei dettagli implementativi si va in contro a una confusione generale data dalla presenza di diverse interpretazioni, spesso errate.

	\item \textbf{Inefficienza:} prima che ci si possa accordare su un valore sono necessari almeno due round.

	\item \textbf{Valore unico:} il consenso viene raggiunto su un unico valore, per tutta la durata vitale del sistema.

	\item \textbf{Convergenza:} non è garantito che l'algoritmo converga, arrivando a una soluzione. L'unica garanzia è che se converge, il valore scelto sarà uno e uno solo.

\end{itemize}
	\section{RAFT Core}
		% -*- root: ../../../main.tex -*-
\subsection{Basics}
		% -*- root: ../../../main.tex -*-
\subsection{Leader Election}
		% -*- root: ../../../main.tex -*-
\subsection{Normal Operation | Log Replication}
		% -*- root: ../../../main.tex -*-
\subsection{Safety e Consistency}
Nelle sezioni precedenti è stato descritto il funzionamento generale di RAFT includendo alcune proprietà e vincoli essenziali per mantenere consistenza tra i log durante le principali operazioni.\\
Qui di seguito ricordiamo le proprietà nominate nelle sezioni precedenti:

\begin{itemize}
	\item{\textbf{Election Safety:}} 
  Per un dato term è possibile al più eleggere \textbf{un solo leader}.
	La dimostrazione di questa proprietà è già stata discussa precedentemente nella sezione \ref{electionsaf}.
	\item{\textbf{Leader Append-Only:}}
  Il leader non cancella \textbf{mai}, ne sovrascrive le entry del proprio log.
  La proprietà è descritta in dettaglio nella sezione \ref{Log Replication}.
	\item{\textbf{Log Matching:}}
  La proprietà di log matching garantisce che:
	\begin{itemize}
		\item{\emph{Se due log hanno un entry con lo stesso index e stesso term, allora quell'entry contiene lo stesso comando(sono identiche)}}.
		\item{\emph{Se due \textit{entry} in log diversi hanno lo stesso index e lo stesso term, allora anche tutte le loro precedenti entry sono identiche tra i due log}}.
	\end{itemize}
	La proprietà è descritta in dettaglio nella sezione \ref{Log Matching}.
\end{itemize}
L'elenco sovrastante non è però completo! Attualmente ci sono casi che non sono gestiti e che possono causare notevole \textbf{inconsistenza} tra i log.\\
Ad esempio se il leader \textit{valida} alcune \textit{entry} mentre un dato follower è offline, nel caso in cui il follower venisse eletto esso potrebbe sovrascrivere le entry gia eseguite con quelle presenti nel proprio log.\\
E' necessario dunque completare l'algoritmo aggiungendo la proprietà chiave di tutte le state machine, la \textbf{State Machine Safety} property. 

  \paragraph{State Machine Safety}
  \emph{Non appena una entry è stata committata ed eseguita in una state machine, allora nessun altra state machine può committare una valore differente per la stessa entry}.\\
  la state machine safety porta all'introduzione di due \textbf{restrizioni} molto importanti all'algoritmo:
  \begin{itemize}
    \item{\textbf{Restrizione durante la leader election}}
    \item{\textbf{Restrizione nei commitment}}
  \end{itemize}


  \subsubsection{Restrizioni nelle Elezioni}
  Durante le elezioni ci sono casi in cui non è possibile determinare se un \textit{entry} è committed oppure no. Come si può vedere in figura \ref{fig:figure10} non è possibile determinare se l'ultima \textit{entry} è stata committed: per questo si aggiunge un \textbf{ulteriore vincolo all'elezione di un candidate}.
  \[
    \emph{\textbf{Scegli il candidato con il log più completo.}}
  \]

  Più precisamente il candidate invia solo le informazioni sull'\textbf{ultima entry} dato che queste informazioni definisco interamente il log.
 \[
      AppendEntries(term, index)
  \]
  Il criterio con cui viene fatta questa valutazione è il seguente:\\
  Sia $lastTerm_v$ il term presente a indice $lastIndex_V$ della entry del log del \textbf{server votante} e $lastTerm_c$ il term presente a indice $lastIndex_C$ della entry del log del \textbf{candidate}, Allora:
  \begin{equation} \label{eq:1}
    \begin{multlined}
    Se:\\
      (lastTerm_v > lastTerm_c) \; \| \;        \\
      (lastTerm_v == lastTerm_c)  \; \&\& \;
      (lastIndex_v > lastIndex_c)
    \implies Reject
    \end{multlined}
  \end{equation}
  In altre parole:
  \begin{itemize}
    \item{\emph{Se l'ultimo term del log del server votante è maggiore di quello del server candidato, allora la  richiesta viene rifiutata. }}
    \item{\emph{Se invece i due ultimi term si equivalgono, la valutazione viene fatta tenendo conto della lunghezza del log:}}
    \begin{itemize}
      \item{\emph{Se l'ultimo indice del log di C è maggiore di quello di V la richiesta viene accettata, altrimenti viene rifiutata.}}
    \end{itemize}
  \end{itemize}
 
  \begin{figure}[H]
  	\centering
  	\includegraphics[width=0.99\columnwidth]{raft/pickingUpToDateLeader}
  	\captionsetup{singlelinecheck=off}
  	\caption[stateDiagramCaption]{
	 In questo caso non è possibile determinare se l'ultima \textit{entry} è validata, inoltre solo il primo server può essere eletto dato che il terzo non è disponibile.}
  	\label{fig:figure10}
  \end{figure}

  \paragraph{Commiting di Entries del term corrente}
  Vediamo ora un esempio di quello che è stato detto fino ad ora. Nella figura \ref{fig:figure11} possiamo vedere che all'indice 4 il leader $S1$ al term 2 è riuscito a \textbf{replicare con successo} la propria entry sulla maggioranza dei server; la entry è dunque \textbf{committed}. A questo punto nel caso in cui il leader $S1$ crashasse, per l'\textbf{election restriction} introdotta sopra, i server $S4$ e $S5$ non potrebbero essere eletti leader!\\
  $S5$ non può essere eletto perché non possiede un term più piccolo di tutti gli altri $\rightarrow$ \textbf{condizione n:1}\\
  $S4$ non può essere eletto poiché anche se possiede un term aggiornato non ha un log completo (più corto) $\rightarrow$ \textbf{condizione n:2} 

  \begin{figure}[H]
    \centering
    \includegraphics[width=0.99\columnwidth]{raft/committingEntryFromCurrentTerm}
    \caption[stateDiagramCaption]{$S4$ e $S5$ si trovano impossibilitati ad essere eletti come leader e quindi non sono in grado di modificare il log di latri server.}
    \label{fig:figure11}
  \end{figure}

  \subsubsection{Committing di Entries di term precedenti}
  Questo caso, rappresentato perfettamente nell'immagine \ref{fig:figure12} rappresenta la situazione peggiore e più inusuale per RAFT; essa si può riassumere come segue:
  \begin{enumerate}
    \item{Il server $S1$, leader al \textbf{term 2}, \textbf{replica la entry} solamente su due server e poi crasha}
    \item{Un altro server, $S5$ viene eletto leader per il \textbf{term 3} da $S3$ $S4$ $S5$, esso appende una serie di entry e poi crasha prima di poter comunicare.}
    \item{Il server $S1$ si risveglia e viene eletto leader da $S1$ $S2$ $S3$ $S4$, e prima di tutto cerca di \textbf{riparare} il log del sever $S3$ inviando la entry mancante.}
    \item{Il server $S5$ quando riceve gli \textbf{ack} da parte dei followers però \textbf{NON COMMITTA!}}
    \item{Se infatti \textbf{$S1$ crashasse dopo aver committato la entry 3} allora $S5$ può essere eletto, soddisfa entrambi i casi di \ref{eq:1}}
    \begin{itemize}
      \item{\textbf{Caso A}}
      \[
        lastTerm_{S5} > lastTerm_{Si} \qquad \forall i \in [2,3,4]
      \]
      \item{\textbf{Caso B}}
      \[
        lastIndex_{S5} > lastIndex_{Si} \qquad \forall i \in [2,3,4]
      \]
    \end{itemize}
    \item{Se $S5$ venisse eletto al \textbf{term 5} cercherebbe di propagare il proprio log e questo \textbf{obbligherebbe $S1$ $S2$ $S3$ $S4$ a dover riscrivere i propri log}.\\
    \textbf{NB:} Qua abbiamo una totale mancanza di safety, infatti \emph{$S1$ non saprebbe cosa fare dato che una volta commitati i log non possono essere cancellati}}.
  \end{enumerate}
  \begin{figure}[H]
    \centering
    \includegraphics[width=0.99\columnwidth]{raft/committingEntryFromEarlierTerm}
    \caption[stateDiagramCaption]{}
    \label{fig:figure12}
  \end{figure}
  \paragraph{Soluzione:}
  Le regole di committing vengono estese al fine di garantire safety. Ogni leader decidere se una entry è da committare in base a queste regole:
  \begin{itemize}
    \item{\emph{\textbf{La entry deve essere presente nella maggioranza dei server}}}
    \item{\emph{\textbf{Almeno una nuova entry proveniente dal leader dell'attuale term deve essere presente nella maggioranza dei server}}}
  \end{itemize}
  Come vediamo in figura \ref{fig:figure13}, nel momento in cui $S1$ viene eletto leader per il \textbf{term 4} e riesce a replicare la entry sulla maggioranza dei servers allora può finalmente \textbf{propagare le informazioni di committment} riferite a term passati.\\
  Questo garantisce che il server $S5$ non possa essere eletto leader al \textbf{term 5} per la regola 1 in \ref{eq:1}.
  \begin{figure}[H]
    \centering
    \includegraphics[width=0.80\columnwidth]{raft/newCommitmentRules}
    \caption[stateDiagramCaption]{Quando $S4$ diviene leader non committa immediatamente le entry passate ma attende l'arrivo di nuove entry. Questo evita che $S5$ possa diventare leader sovrascrivendo i logs e portando inconsistenze.}
    \label{fig:figure13}
  \end{figure}
  \subsubsection{Neutralizzazione di vecchi leader}
  Supponiamo che un leader venga momentaneamente disconnesso dalla rete, creando una cosiddetta \textbf{network partition}. RAFT essendo un algoritmo che soddisfa il requisito di \textbf{partition tolerance} garantisce che il cluster sopravviva alla failure di un membro; il cluster semplicemente eleggerà un nuovo leader.\\
  \emph{Cosa succede però se il vecchio leader torna in campo e decide di continuare a governare?}\\
  Per evitare inconsistenze RAFT mette in pratica un comportamento molto semplice, chiamato \textbf{Neutralization of stale leader}, esso funziona in modo molto semplice:
  \begin{itemize}
    \item{Ogni messaggio scambiato contiene al suo interno il \textbf{term del mittente}}.
    \item{\textbf{Caso A}}
      \[
        senderTerm > receiverTerm
      \]
      \begin{itemize}
        \item{\textbf{Il messaggio viene rifiutato.}}
        \item{\textbf{Il mittente si converte a follower.}}
      \end{itemize}
      \item{\textbf{Caso B}}
      \[
        receipTerm > senderTerm
      \]
      \begin{itemize}
        \item{\textbf{Il ricevente si converte a follower.}}
        \item{\textbf{Il ricevente aggiorna il suo term a l'ultimo ricevuto.}}
        \item{\textbf{Il ricevente processa il messaggio normalmente.}}
      \end{itemize}
  \end{itemize}
  Seguendo questi semplici casi, una volta che una \textbf{lezione} per un nuovo leader è \textbf{terminata} non è possibile che un leader deposto possa inviare messaggi alla maggioranza del cluster poiché tale maggioranza possiederà un \textbf{term aggiornato dall'ultima elezione}.
	\section{RAFT Extensions}
		% -*- root: ../../../main.tex -*-
\subsection{Client Protocol}
		% -*- root: ../../../main.tex -*-
\subsection{Configuration Changes}
La configurazione del sistema può cambiare nel tempo, ad esempio quando una macchina va incontro a crash, viene sostituita o viene aggiunta al cluster.\\

Le informazioni sulla configurazione del sistema sono cruciali perchè sapere \textbf{quanti} e \textbf{quali} server fanno parte del cluster è necessario per operazioni che necessitano di decidere basandosi sulla maggioranza, come l'elezione del leader o il commit di una entry.\\

\subsubsection{Contemporanea presenza di due configurazioni}
  Il problema che si incontra in tutti i sistemi distribuiti è che \textit{la \textbf{transizione} tra una configurazione e l'altra \textbf{non avviene nello stesso istante} per tutte le macchine}. Esisterà dunque un tempo \textbf{t} per cui \textbf{alcune} macchine saranno aggiornate sulla \textbf{nuova configurazione} mentre \textbf{altre} saranno ancora nella \textbf{vecchia}. Ciò porta a una situazione in cui è possibile che ci siano due gruppi di server, che si trovano in \textbf{due diverse configurazioni} e che rappresentano la \textbf{maggioranza} per quella configurazione.\\

  \begin{figure}[H]
    \centering
    \includegraphics[width=0.90\columnwidth]{raft/configChanges.pdf}
    \caption{Esempio di \textbf{maggioranze diverse in due diverse configurazioni}: i server \textit{S4} e \textit{S5} entrano a far parte del cluster, ma al tempo \textbf{t}, si ha che \textit{S1} e \textit{S2} sono ancora alla \textbf{vecchia configurazione}, mentre \textit{S3}, \textit{S4} e \textit{S5} sono \textbf{passati alla nuova}.
    \textit{S1} e \textit{S2} costituiscono una maggioranza, nella configurazione vecchia (sono \textbf{2 su 3}), ma anche \textit{S3}, \textit{S4} e \textit{S5} possono formare una maggioranza nella loro configurazione di riferimento (\textbf{3 su 5}). }
    \label{fig:figure 8}
  \end{figure}

  \subsubsection{Problema}
    Sulla base di queste \textbf{maggioranze} possono essere portate a termine delle operazioni, come ad esempio, il commit  di una entry. Questo potrebbe generare delle \textbf{inconsistenze}, come nel caso seguente:
  \begin{enumerate}
    \item{\emph{I server \textbf{S1} e \textbf{S2} formano una \textbf{maggioranza} nella \textbf{vecchia} configurazione}}
    \item{\emph{I server \textbf{S3}, \textbf{S4} e \textbf{S5} formano una \textbf{maggioranza} nella \textbf{nuova} configurazione}}
    \item{\emph{la \textbf{maggioranza} data da S1 e S2 decide di fare \textbf{commit} dell'entry \textbf{e1}}}
    \item{\emph{la \textbf{maggioranza} data da S3,S4 e S5 decide di fare \textbf{commit} dell'entry \textbf{e2!=e1}}}
    \item{\emph{in corrispondenza dello \textbf{stesso indice}, si avranno dei \textbf{log diversi} che porteranno all'esecuzione di istruzioni diverse, generando \textbf{inconsistenza}}}
  \end{enumerate}

  Per risolvere questo problema, il passaggio tra una configurazione e l'altra non è immediato, ma viene fatto in \textbf{due fasi}.
    


  \subsubsection{Soluzione}   

    Per cambiare la configurazione del sistema, si ricorre a una \textbf{fase intermedia} posta tra la vecchia configurazione e quella nuova, in maniera tale da \textbf{evitare} di incorrere in \textbf{inconsistenze}.
    Durante questa fase intermedia, chiamata \textbf{joint consensus}, per prendere le decisioni si tengono in considerazione entrambe le maggioranze: sia quella della nuova configurazione (\textit{C\_new}) che quella della vecchia (\textit{C\_old}) in maniera tale che nè \textit{C\_old} nè \textit{C\_new} possano prendere \textbf{decisioni unilaterali}.\\

    Le caratteristiche di questo approccio sono le seguenti:
    
    \begin{itemize}
      \item{\textbf{Cambiamento della configurazione:} il passaggio a una nuova configurazione viene innescato da una richiesta. Quando il leader riceve questa richiesta la aggiunge al proprio log e la propaga, come farebbe normalmente, ma l'azione ha \textbf{effetto immediato} poichè il leader applica il cambiamento di configurazione il prima possibile, \textbf{senza attendere il commit}.}

      \item{\textbf{Configurazione intermedia:} successivamente, il leader, passa alla configurazione intermedia (\textit{C\_old+new}) e prende tutte le successive decisioni basandosi su di essa.\\
      Ad esempio, per capire se è possibile procedere con il commit di una entry, verifica che essa abbia la \textbf{\textit{maggioranza}} \textbf{sia nella vecchia} configurazione \textbf{che in quella nuova}.} 

      \item{\textbf{Decisioni sulla base di \textit{C\_old} o \textit{C\_old+new}:} esiste un periodo di tempo in cui la \textbf{entry} della nuova configurazione è \textbf{presente} nel log del leader ma \textbf{non} è ancora stata \textbf{committata}. In questo lasso di tempo è possibile che le \textbf{decisioni} possano essere prese sotto \textbf{entrambe le configurazioni}.\\ 

      Ad esempio, se il \textbf{leader} decade prima di aver \textbf{replicato} l'entry della \textbf{nuova configurazione} negli altri log, è possibile che venga eletto come \textbf{nuovo leader} un server che ha ancora la \textbf{vecchia configurazione}.\\

      Tuttavia, prima o poi arriverà un leader che non fallirà anzitempo e la nuova configurazione verrà committata. } 

      \item{\textbf{Joint consensus:} una volta fatto il \textbf{commit} diventa \textbf{impossibile} che vengano prese decisioni solo sulla base di \textit{C\_old}, perchè ora il sistema è interamente sotto la configurazione \textit{\textbf{C\_old+new}}, in uno stato di \textit{joint consensus}.
      Durante il joint consensus, la \textbf{nuova configurazione} può essere \textbf{aggiunta} al log e \textbf{propagata}.} 


      \item{\textbf{Decisioni sulla base di \textit{C\_old+new} o \textit{C\_new}:} esiste un periodo di tempo tra l'\textbf{aggiunta} nel log del leader della \textbf{nuova configurazione} e il \textbf{commit} della stessa, in cui le \textbf{decisioni} possono essere prese sulla base della configurazione \textit{\textbf{C\_old+new}} o sulla base della \textbf{nuova configurazione}. \\
      Ciò accade perchè nel caso in cui il \textbf{leader} sia soggetto a \textbf{crash}, \textbf{prima} che abbia proceduto al \textbf{commit} dell'entry relativa alla \textbf{nuova} configurazione, può essere eletto un \textbf{nuovo leader} che ha ancora la \textbf{configurazione intermedia}.\\
      Tuttavia, come nel caso precedente, si ha che \textbf{prima o poi} un leader riuscirà a fare \textbf{commit della nuova configurazione} e, da quel momento in poi, le decisioni verranno prese \textbf{solo} sulla base di quest'ultima.} 

      \item{\textbf{Elezione di leader in \textit{C\_new+old} durante\textit{ C\_new}:} non c'è \textbf{nessun momento} in cui sia \textit{C\_old} che \textit{C\_new} possono prendere \textbf{decisioni unilaterali}, generando \textbf{conflitti}. Tuttavia è possibile che anche \textbf{dopo} che \textit{C\_new} è stata \textbf{committata}, venga eletto un \textbf{leader} che \textbf{non è ancora} in tale configurazione e che quindi \textbf{non può prendere decisioni}. In questo caso esso dovrà \textbf{``dimettersi''} dando vita ad una \textbf{nuova elezione} nel momento in cui il timeout di uno degli altri follower scadrà. } 

    \end{itemize}

  \begin{figure}[H]
    \centering
    \includegraphics[width=0.90\columnwidth]{raft/jointConsensus.pdf}
    \caption{Linea temporale che mostra le fasi del passaggio del sistema da una configurazione all'altra.}
    \label{fig:figure 9}
  \end{figure}


		% -*- root: ../../../main.tex -*-
\subsection{Log Compaction}