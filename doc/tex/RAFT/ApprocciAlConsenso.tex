% -*- root: ../../main.tex -*-
Esistono generalmente due approcci al problema del consenso:
\begin{itemize}
  \item{\textbf{Symmetric | Leader-less:}}
  in questa tipologia i server hanno tutti lo \textbf{stesso ruolo} e tutti lo stesso potere decisionale in qualsiasi momento.\\
  I \textbf{client} possono quindi \textbf{contattare} e fare richiesta verso \textbf{qualunque server} presente all'interno del cluster. 
  \item{\textbf{Asymmetric | Leader-base:}}
  in qualsiasi momento i server non hanno \textbf{mai} lo \textbf{stesso ruolo} e lo stesso potere decisionale. Esiste infatti sempre un server che si trova \textbf{in carica}. Questo particolare server viene denominato \textbf{leader}. Il leader gestisce solitamente tutte le operazioni del cluster, mentre gli altri server sono denominati \textbf{servant} e portano a compimento le richieste del leader.\\
  In questo tipo di sistema i \textbf{client} possono \textbf{comunicare solamente con il leader}; nel caso in cui un client contatti un server diverso dal leader viene immediatamente \textbf{rediretto} al leader corrente.
\end{itemize}
Vedremo che RAFT implementa la seconda versione di consenso; questo garantisce all'algoritmo di essere estremamente più semplice dei suoi predecessori leader-less garantendo comunque lo stesso grado di safety.